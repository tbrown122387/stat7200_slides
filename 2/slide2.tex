%\documentclass{beamer}
\documentclass[handout]{beamer}
% \usepackage{beamerthemesplit} // Activate for custom appearance

\usepackage{mathtools}
\DeclarePairedDelimiter{\ceil}{\lceil}{\rceil}

\usetheme{Boadilla}
\newtheorem{Proposition}[theorem]{Proposition}

\setbeamertemplate{theorems}[numbered]

\title{STAT 7200}
 \subtitle{Introduction to Advanced Probability \newline Lecture 2}
\author{Taylor R. Brown}
\institute{}
\date{}

\begin{document}

\frame{\titlepage}

\section[Outline]{}
\frame{\tableofcontents
``A First Look at Rigorous Probability Theory" (Jeffrey Rosenthal) Sections A.3 and A.4
}



\section{Mathematical Background}


\subsection{Limits}
\subsubsection{Limits of Sequences of Real Numbers}

\frame
{
  \frametitle{Limits: Limits of Sequences of Real Numbers}

  \begin{itemize}
  \item <1->  \textbf{Limit of A Sequence of Real Numbers} A sequence of real numbers $x_1,x_2,\cdots,x_n,\cdots $ converges to another real number $x$ if, given any $\varepsilon>0$, there is a $N\in\mathbf{N}$, so that for any $n>N$, $|x_n-x|<\varepsilon$. We denote this as $\lim_{n\rightarrow\infty} x_n=x$.
  \item <2->  Intuition: imagine a small interval (or ball) centered at $x$ with radius $\varepsilon$. As long as the $x$ is the limit of the sequence, no matter how small you choose $\varepsilon$, there will only be a finite number of $x_n$ outside of this interval.
  
      \item<3-> \textbf{Example} Show that $\lim_{n\rightarrow\infty}  \frac{1}{n^k}= 0.$

    \item<4-> \textbf{Proof} First, choose an arbitrary $\varepsilon>0$.     
    \item[]<5-> Set $N := \ceil[\big]{\frac{1}{\varepsilon^{1/k}}}$. Then $n > N$ guarantees  $|\frac{1}{n^k}-0|<\varepsilon$. 
    
    
  \end{itemize}
}

\subsubsection{Sequences that Converge to Infinity and Sequences without Limits}

\frame
{
  \frametitle{Sequences that Converge to Infinity and Sequences without Limits}

  \begin{itemize}
  \item <1->  \textbf{Converges to Infinity} A sequence of real numbers $x_1,x_2,\cdots,x_n,\cdots $ converges to infinity if for any $M\in\mathbf{R}$, there is a $N\in\mathbf{N}$, so that for any $n>N$, $X_n>M$. We denote this as $\lim_{n\rightarrow\infty} x_n=\infty$. We can also define the convergence to negative infinity in a similar fashion. 
  
      \item<2-> \textbf{Example} $\lim_{n\rightarrow\infty} n= \infty.$

    \item<3-> There are also sequences that do not have a finite or infinite limit. For instance, the sequence $0,1,0,1,0,1,\cdots$ oscillates between $0$ and $1$. Thus It does not converge to either $1$ or $0$. 
 
    \item<4-> Still, for any real sequence, there is always at least a subsequence that converges to a finite value or infinity. This is reflected in the following facts: 1) For any sequence, you can always find a monotone subsequence, which converges to a finite value or infinity. 2) Bolzano-Weierstrass theorem:  if the original sequence is bounded, you will always be able to find a subsequence that converges (to a finite value). 
         
  \end{itemize}
}



\subsubsection{Bounds of Limit}

\frame
{
  \frametitle{Bounds and Limits}

  \begin{itemize}
  \item <1->   A set $A\subset R$ is \textbf{bounded above (or below) } if there is a real number $M$ such that $a\leq M$ (or $a\geq M$ ) for all $a\in A$.  A set that is bounded above and blow is called \textbf{bounded}. 
  
  \item [] <2-> \begin{Proposition} If $\lim_{n\rightarrow\infty} x_n=x$, then the set $\{x_n:n\in \mathbf{N}\}$ is bounded. \end{Proposition} 
    \item<3-> \textbf{Proof} Choose $\varepsilon=1$. Because $\lim_{n\rightarrow\infty} x_n=x$, we can find a large $N$ where $|x_n-x|< 1$ for any $n>N$.
    \item[]<4-> Let $M=\max \{x_1,x_2,\cdots, x_N, x+1\}$, $L=\min \{x_1,x_2,\cdots, x_N, x-1\}$. Clearly $L\leq x_n\leq M$ for all $n\in \mathbf{N}$.
    

         
  \end{itemize}
}


\subsubsection{Properties of Limits}

\frame
{
  \frametitle{Properties of Limits}

   \begin{itemize}
  \item [] <1-> \begin{Theorem} If $\lim_{n\rightarrow\infty} x_n=x$, and $\lim_{n\rightarrow\infty} y_n=y$, then 
  
  1) For any $a$, $\lim_{n\rightarrow\infty} a x_n=ax$; 2) $\lim_{n\rightarrow\infty} (x_n+y_n)=x+y$; 
  
  3) $\lim_{n\rightarrow\infty} (x_ny_n)=xy$; 4) If $x>0$, then $\lim_{n\rightarrow\infty} \frac{1}{x_n}=\frac{1}{x}$.
   \end{Theorem} 
    
      \item<2-> \textbf{Proof} We only consider the situation in which both limits are finite. 

    \item[]<3->  2): By definition, given any $\varepsilon>0$, there are $N_1, N_2\in \mathbf{N}$, so that $|x_n-x|<\varepsilon/2$ for $n>N_1$ and $|y_n-y|<\varepsilon/2$ for $n>N_2$.
        \item[]<4-> Now we let $N^*=\max(N_1, N_2)$, then for any $n>N^*$, $|x_n-x|<\varepsilon/2$ \textit{ and } $|y_n-y|<\varepsilon/2$.

        \item[]<4-> Furthermore, for $n>N^*$, we have, $|x_n+y_n-x-y|\leq |x_n-x|+|y_n-y|<\varepsilon/2+\varepsilon/2=\varepsilon$. 
        
        Thus, $\lim_{n\rightarrow\infty} (x_n+y_n)=x+y$.
        
        
  \end{itemize}
}


\frame
{
  \frametitle{Properties of Limits (continued)}

   \begin{itemize} \item<1-> 3) $\lim_{n\rightarrow\infty} (x_ny_n)=xy$
    
      \item<2-> \textbf{Proof:} The intuition is to show $|x_ny_n-xy|$ can be arbitrarily small for large enough $n$. This can be shown by the following inequality: $|x_ny_n-xy|=|x_ny_n -xy_n+xy_n-xy|\leq |y_n||x_n-x|+|x| |y_n-y|$, in which $|y_n|$ approaches $y$, and $|x_n-x|$, $|y_n-y|$ approaches 0 for large $n$. A rigorous proof for the case $x\neq 0$ is shown below:
      
        \item[]<4-> a)  For any $\varepsilon>0$, there is $N_1\in \mathbf{N}$ so that for any $n>N_1$, $|y_n-y|<\varepsilon/(2|x|)$.
        \item[]<5-> b) Choose any constant $\delta>0$. Then there is $N_2\in \mathbf{N}$ so that for any $n>N_2$, $|y_n-y|<\delta$, which further implies $|y_n|<|y|+\delta$. 
        \item[]<6-> c) For the same $\varepsilon>0$, there is $N_3\in \mathbf{N}$ so that for any $n>N_3$, $|x_n-x|<\varepsilon/(2(|y|+\delta))$.
        \item[]<7-> d) Now we let $N^*=\max(N_1, N_2,N_3)$, then for any $n>N^*$,       
       \begin{align*}|x_ny_n-xy|& =|x_ny_n -xy_n+xy_n-xy|\leq |y_n||x_n-x|+|x| |y_n-y| \\ & \leq (|y|+\delta) \varepsilon/(2(|y|+\delta))+|x| \varepsilon/(2|x|)=\varepsilon\end{align*}

        
        Thus, $\lim_{n\rightarrow\infty} (x_ny_n)=xy$.
        
        
  \end{itemize}
}


\subsubsection{Sums of Infinite Sequences}

\frame
{
  \frametitle{Sum of Infinite Sequences}

   \begin{itemize}
  \item <1-> For sequence $x_1,x_2,\cdots,x_n,\cdots $, we define its sum as 
  
  $$\sum_{n=1}^{\infty} x_n=\lim_{n\rightarrow\infty} \sum_{i=1}^n x_i $$
  
   \item<2-> For nonnegative sequences, the limit is either finite or infinity.    
   \item<3-> \textbf{Examples} $$\sum_{n=1}^{\infty} \frac{1}{n}= \infty;\ \sum_{n=1}^{\infty} \frac{1}{n!}= e.$$
       
  \end{itemize}
}


\subsubsection{On Sums of Infinite Sequences}

\frame
{
  \frametitle{Sums of Infinite Sequences}

   \begin{itemize}
  \item <1-> Recall that the sum of an infinite sequence $x_1,x_2,\cdots,x_n,\cdots $ is:
  
  $$\sum_{n=1}^{\infty} x_n=\lim_{n\rightarrow\infty} \sum_{i=1}^n x_i $$
  
 $\sum_{n=1}^{\infty} x_n$ converges if the limit of the partial sum is finite. 
 
  
   \item[]<2-> \begin{Theorem} 1) If $\sum_{n=1}^{\infty} x_n$ converges, then for every $\varepsilon>0$, there is an $N\in \mathbf{N}$ so that $|\sum_{k=n+1}^{\infty} x_k|<\varepsilon$ for all $n>N$ \newline 2) Let $\{x_{n}\}_{n\in \mathbf{N}}$ and $\{y_{n}\}_{n\in \mathbf{N}}$ be two sequences of real numbers with $|x_n|<y_n$ for all $n$. If $\sum_{n=1}^{\infty} y_n$ converges, then $\sum_{n=1}^{\infty} x_n$ also converges and $|\sum_{n=1}^{\infty} x_n|<\sum_{n=1}^{\infty} y_n$ \end{Theorem} 
      
       
  \end{itemize}
}


\subsubsection{More on Limits: Squeeze Theorem}

\frame
{
  \frametitle{Squeeze Theorem}

  \begin{itemize}
  \item [] <1-> \begin{Theorem}Suppose that we have three sequences $\{a_n\}$, $\{b_n\}$, $\{c_n\}$ that satisfy $a_n\leq b_n\leq c_n$ for all $n$ and $\lim_{n\rightarrow \infty} a_n=\lim_{n\rightarrow \infty} c_n=L.$ Then $\lim_{n\rightarrow \infty} b_n=L$
  \end{Theorem}
  
  \item<2-> \textbf{Proof} For any $\varepsilon>0$, there are $N_1, N_2\in \mathbf{N}$, so that $|a_n-L|<\varepsilon$ for $n>N_1$ and $|c_n-L|<\varepsilon$ for $n>N_2$. 
        \item[]<3-> Now we let $N^*=\max(N_1, N_2)$, then for any $n>N^*$, $|a_n-L|<\varepsilon$ \textit{ and } $|c_n-L|<\varepsilon$. These two inequalities further imply $L-\varepsilon <a_n\leq b_n\leq c_n<L+\varepsilon$. 

        \item[]<4-> Thus, for $n>N^*$, $|b_n-L|<\varepsilon$. We have $\lim_{n\rightarrow \infty} b_n=L$
        

  \item <5->  \textbf{Example} $\lim_{n\rightarrow \infty} \frac{\sin n}{n}=0$ since $ -\frac{1}{n}\leq \frac{\sin n}{n}\leq \frac{1}{n}$  
             
  \end{itemize}
}



\subsubsection{Limits Preserve Order}

\frame
{
  \frametitle{Limits Preserve Order}

  \begin{itemize}
  \item [] <1-> \begin{Theorem}Suppose that we have two sequences $\{a_n\}$, $\{b_n\}$ that satisfy $a_n\leq b_n$ for all $n$. If $\lim_{n\rightarrow \infty} a_n=L$ and $\lim_{n\rightarrow \infty} b_n=M.$ Then $L\leq M$.
  \end{Theorem}
  
  \item<2-> \textbf{Proof} Assume to the contrary that $L>M$. Pick $\varepsilon>0$ such that $M+\varepsilon< L-\varepsilon$ (e.g. $\varepsilon=(L-M)/4$)
  
     \item[]<3-> For this same $\varepsilon>0$, pick $N\in\mathbf{N}$ so that $|a_n-L|<\varepsilon$  and $|b_n-M|<\varepsilon$ for $n>N$.  However, these two inequalities imply $a_n>L-\varepsilon>M+\varepsilon>b_n$ when $n>N$, which contradicts the hypothesis that $a_n\leq b_n$ for all $n$. Thus, $L\leq M$. 
                  
  \end{itemize}
}





\subsubsection{Supremum and Infimum}

\frame
{
  \frametitle{Supremum and Infimum}

  \begin{itemize}
  \item <1->  \textbf{Supremum} For any nonempty subset $A$ of $\mathbf{R}$ that is bounded above, the \textbf{supremum} or \textbf{least upper bound} is the number $L$ so that \newline 1) $a\leq L$ for all $a\in A$. 2) For any other upper bound $L'$ of $A$, $L'\geq L$. The supremum of $A$ is denoted by $\sup A$.

  \item <2->  \textbf{Infimum} Similarly, we can also define the \textbf{infimum} or \textbf{greatest lower bound} for any nonempty subset $A$ of $\mathbf{R}$ that is bounded below as $\inf A$ 
  
        \item<3-> \textbf{Example} \newline1) $\inf \{0,1,2,3,\cdots,n\cdots\}=0;$  \newline 2) $\sup \{1/2,2/3,3/4,\cdots,n/(n+1), \cdots\}=1.$

        \item<4-> \textbf{Exercise} Show that, if $A$ and $B$ are two nonempty subset of $\mathbf{R}$, $A\subset B$, and if the corresponding suprema and infima exist, then $\sup A\leq \sup B$ and $\inf A\geq \inf B$.
        

        
  \end{itemize}
}



\frame
{
  \frametitle{Properties of Supremum and Infimum}

  \begin{itemize}
  
\item<1-> Every nonempty subset of $\mathbf{R}$ that is bounded above has a supremum. Similarly, every nonempty subset $\mathbf{R}$ that is bounded below has an infimum.

  
     \item <2-> If a nonempty set $A$ is not bounded below, we will denote $\inf A=-\infty$. Similarly, if $A$ is not bounded above, $\sup A=\infty$. 
   \item [] <3-> \begin{Proposition} If $A$ is a non-empty set that is bounded below. Then for any $\varepsilon>0$, there is $a\in A$ with $\inf A\leq a<\inf A+\varepsilon$\end{Proposition} 

  \item <4->  \textbf{Proof} If such $a$ does not exist, then for all $a\in A$, we have $a\geq \inf A+\varepsilon$. That is, $\inf A+\varepsilon$ is a lower bound of $A$. However, by definition $\inf A$ is the greatest lower bound of $A$ and we reach a contradiction.   
        
  \end{itemize}
}



\subsubsection{Monotone Convergence Theorem}

\frame
{
  \frametitle{Monotone Convergence Theorem}

  \begin{itemize}
  \item [] <1-> \begin{Theorem} A monotone increasing sequence that is bounded above converges (to a finite value). A monotone decreasing sequence that is bounded below converges (to a finite value). \end{Theorem} 

    \item<2-> \textbf{Proof} Suppose that sequence $x_1,x_2,\cdots,x_n,\cdots $ is a monotone increasing sequence that is bounded above, and denote $L=\sup\{x_n: n\in \mathbf{N}\}$. We will show that $\lim_{n\rightarrow\infty} x_n=L$.

    \item[]<3->  For any $\varepsilon>0$, since $L-\varepsilon$ can not be an upper bound of $\{x_n\}$, there must be a natural number $N$ so that $x_N>L-\varepsilon$. 
        \item[]<4-> However, since $\{x_n\} $ is a increasing sequence, for all $n>N$, $L\geq x_n\geq x_N>L-\varepsilon$. 
        
         \item[]<5->  The inequality above suggests that $|x_n-L|<\varepsilon$ for all $n>N$. Thus, $\lim_{n\rightarrow\infty} x_n=L$.         
  \end{itemize}
}



\subsubsection{Limit Superior and Limit Inferior}

\frame
{
  \frametitle{Limit Superior and Limit Inferior}

   \begin{itemize}
  \item<1-> \textbf{Limit Superior and Limit Inferior} For $x_1,x_2,\cdots,x_n,\cdots $, the \textbf{limit inferior} is defined as $\liminf_{n\rightarrow \infty} x_n=\lim_{n\rightarrow \infty} (\inf_{m\geq n} x_m)$  the  \textbf{limit superior} is defined as $\limsup_{n\rightarrow \infty} x_n=\lim_{n\rightarrow \infty} (\sup_{m\geq n} x_m)$
  
      \item<2-> \textbf{Exercise} Find the limit superior and limit inferior for $0,1,0,1,\cdots$?
  
    \item<3-> Both limit superior and limit inferior exist (maybe infinity). For this, note that both $\{v_n: v_n=\inf_{m\geq n} x_m\}$ and $\{u_n: u_n=\sup_{m\geq n} x_m\}$ are monotone squences. 
      
   \item[]<4-> \begin{Proposition}  $\inf_{n} x_n \leq \liminf_{n\rightarrow \infty} x_n\leq \limsup_{n\rightarrow \infty} x_n\leq \sup_{n} x_n$\end{Proposition} 
   

  \end{itemize}
}

\subsubsection{Limit Superior, Limit Inferior and Limit}

\frame
{
  \frametitle{Limit Superior, Limit Inferior and Limit}

   \begin{itemize}
   
      \item[]<1-> \begin{Theorem}  $\lim_{n\rightarrow\infty}x_n$ exists if and only if $\liminf_{n\rightarrow \infty} x_n=\limsup_{n\rightarrow \infty} x_n$\end{Theorem} 
  \item<2-> \textbf{Proof} Let $\{v_n: v_n=\inf_{m\geq n} x_m\}$ and $\{u_n: u_n=\sup_{m\geq n} x_m\}$, then $\liminf_{n\rightarrow \infty} x_n=\lim_{n\rightarrow \infty} v_n$ and $\limsup_{n\rightarrow \infty} x_n=\lim_{n\rightarrow \infty} u_n$. Note that for all $n$, we have $v_n\leq x_n\leq u_n$.
   \item[]<3-> 1) ``if" part: By the Squeeze Theorem, \newline if $\lim_{n\rightarrow \infty} v_n=\lim_{n\rightarrow \infty} u_n=x$, we must have $\lim_{n\rightarrow\infty}x_n=x$. 
     \item[]<4-> 2) ``only if" part: If $\lim_{n\rightarrow\infty}x_n=x$, then for any $\varepsilon$, there is a $N\in \mathbf{N}$, so that for $n>N$, $x-\varepsilon<x_n<x+\varepsilon$. 
     \item[]<5-> Consequently, we deduce that, for $n>N$, $x-\varepsilon\leq v_n\leq u_n\leq x+\varepsilon$. Thus, $x-\varepsilon \leq \lim_{n\rightarrow \infty} v_n\leq \lim_{n\rightarrow \infty} u_n\leq x+\varepsilon$. Furthermore, since $\varepsilon$ is arbitrary. we must have $x\leq \lim_{n\rightarrow \infty} v_n\leq \lim_{n\rightarrow \infty} u_n\leq x$. Thus, $\liminf_{n\rightarrow \infty} x_n=\limsup_{n\rightarrow \infty} x_n=x$.
     
 
       
  \end{itemize}
}






\frame
{
  \frametitle{Example}

   \begin{itemize}
  \item <1-> \textbf{Problem:}  Let $\{x_{n}\}_{n\in \mathbf{N}}$ and $\{y_{n}\}_{n\in \mathbf{N}}$ be two sequences of real numbers with $y_n\geq 0$ for all $n$ so that $\limsup_{n\rightarrow \infty} \frac{|x_n|}{y_n} <\infty$ and $\sum_{n=1}^{\infty} y_n<\infty$, then  $\sum_{n=1}^{\infty} x_n$ converges .
  
   \item<2-> \textbf{Proof:}  The key here is to show that $|x_n|$ is bounded by $y_n$ times a positive constant. 
   \item[]<3-> Since  $\limsup_{n\rightarrow \infty} \frac{|x_n|}{y_n}=\lim_{n\rightarrow \infty} (\sup_{m\geq n} \frac{|x_m|}{y_m} )$ converges, $\sup_{ n} \frac{|x_n|}{y_n}$ must be finite and positive. 
      \item[]<4-> Assuming that $\sup_{ n} \frac{|x_n|}{y_n}=M>0$, then for any $n$, $\frac{|x_n|}{y_n}\leq M$, and $|x_n|\leq M y_n$. 

      \item[]<5-> However, as $\sum_{n=1}^{\infty} y_n<\infty$, $\sum_{n=1}^{\infty} M y_n$ is also finite. That is, $|x_n|$ is bounded by a sequence whose sum converges, then $\sum_{n=1}^{\infty} x_n$ also converges and $|\sum_{n=1}^{\infty} x_n|\leq M \sum_{n=1}^{\infty}  y_n$.   
         
       
  \end{itemize}
}




\subsubsection{Exchange Summation and Limit}

\frame
{
  \frametitle{Exchange Summation and Limit}

   \begin{itemize}
   
      \item[]<1-> \begin{Theorem} Let $\{x_{nk}\}_{n,k\in\mathbf{N}}$ be a collection of real numbers, so that $\lim_{n\rightarrow \infty} x_{nk}=a_k$ for each fixed $k$. If $\sum_{k=1}^{\infty} \sup_n |x_{nk}|<\infty$, then $\lim_{n\rightarrow \infty } \sum_{k=1}^{\infty} x_{nk}=\sum_{k=1}^{\infty} a_k = \sum_{k=1}^{\infty} \lim_{n\rightarrow \infty } x_{nk}$\end{Theorem} 
 
  \item<2-> \textbf{Proof} For any fixed $k$, $|a_k|=|\lim_{n\rightarrow \infty} x_{nk}|  \leq \sup_n |x_{nk}|$, so $\sum_{k=1}^n |a_k|<\infty$.
  
  
  
     \item[]<3-> We now need to prove that $ |\sum_{k=1}^{\infty} x_{nk}- \sum_{k=1}^{\infty} a_k|=  |\sum_{k=1}^{\infty} ( x_{nk}- a_k)| $ is smaller than any $\varepsilon>0$ for large $n$. To achieve this, we should break this sum into two parts: $ |\sum_{k=1}^{\infty} ( x_{nk}- a_k)| \leq |\sum_{k=1}^{K} ( x_{nk}- a_k) |+|\sum_{k=K+1}^{\infty} ( x_{nk}- a_k)| $.
     
     \item[]<4-> 1) For the second sum, note that $|\sum_{k=K+1}^{\infty} ( x_{nk}- a_k)| \leq \sum_{k=K+1}^{\infty} | x_{nk}- a_k| \leq 2 \sum_{k=K+1}^{\infty} \sup_n |x_{nk}| $. However, since $\sum_{k=1}^{\infty} \sup_n |x_{nk}|<\infty$, we should be able to choose $K$ big enough so that $\sum_{k={K+1}}^{\infty} \sup_n |x_{nk}|<\varepsilon/4$.
     
    
 
       
  \end{itemize}
}


\frame
{
  \frametitle{Exchange Sum and Limit: continued}

   \begin{itemize}
   
 
  \item<1-> \textbf{Proof: continued} Our goal is to show that $ |\sum_{k=1}^{\infty} ( x_{nk}- a_k)|  \leq |\sum_{k=1}^{K} ( x_{nk}- a_k) |+|\sum_{k=K+1}^{\infty} ( x_{nk}- a_k)| <\varepsilon $ for big $n$, and we have already proved that we can choose $K$ big enought so that $|\sum_{k=K+1}^{\infty} ( x_{nk}- a_k)|<\varepsilon/2$. 
  
  
     \item[]<2-> 2) For the first sum, since $ |\sum_{k=1}^{K} ( x_{nk}- a_k)| \leq  \sum_{k=1}^{K} |( x_{nk}- a_k)|$, and $\lim_{n\rightarrow \infty} x_{nk}=a_k$. Then for each $1\leq k\leq K$, we can find $N_K \in \mathbf{N}$ so that for $n>N_k$, $|x_{nk}-a_k|<\varepsilon/(2K)$.
     
       \item[]<3->  If we choose $N^*=\max(N_1,N_2,\cdots, N_K)$, then for all $n>N^*$, $\sum_{k=1}^{K} |( x_{nk}- a_k)|<\sum_{k=1}^K \varepsilon/(2K) =\varepsilon/2$.
     
       \item[]<4-> 3)  Now combine the results in both 1) and 2), we conclude that  $ |\sum_{k=1}^{\infty} ( x_{nk}- a_k)| <\varepsilon$ for $n>N^*$. Thus, $\lim_{n\rightarrow \infty } \sum_{k=1}^{\infty} x_{nk}=\sum_{k=1}^{\infty} a_k= \sum_{k=1}^{\infty} \lim_{n\rightarrow \infty } x_{nk}$. That is, the exact order of taking limit with respect to $n$ and summing over $k$ does not matter. 
        
       
  \end{itemize}
}



\end{document}
