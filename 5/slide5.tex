%\documentclass{beamer}
\documentclass[handout]{beamer}
% \usepackage{beamerthemesplit} // Activate for custom appearance
\usetheme{Boadilla}
\newtheorem{Proposition}[theorem]{Proposition}%
\newcommand{\BP}{\mathbf{P}}

\setbeamertemplate{theorems}[numbered]

\title{STAT 7200}
 \subtitle{Introduction to Advanced Probability \newline Lecture 5}
\author{Taylor R. Brown}
\institute{}
\date{}

\begin{document}

\frame{\titlepage}

\section[Outline]{}
\frame{\tableofcontents
``A First Look at Rigorous Probability Theory" (Jeffrey Rosenthal) Section 2.3 (continued)
}


\section{Probability Triple}

\subsubsection{Extension Theorem}
\frame
{
  \frametitle{Extension Theorem}

   \begin{itemize}

             \item<1-> [] \begin{Theorem} \textbf{The Extension Theorem}  Let $\mathcal{J}$ be a semialgebra of subsets of $\Omega$,  $\mathbf{P}$ a function from $\mathcal{J}$  to [0,1] with the following properties:
             \newline
             
             a) $\mathbf{P} (\emptyset)=0, \mathbf{P}(\Omega)=1$.
                 \newline
         
             b) $\BP(\bigcup_{i=1}^k A_i)\geq \sum_{i=1}^k \BP(A_i)$ whenever $A_1,\ldots,A_k \in\mathcal{J}$, $\bigcup_{i=1}^k A_i  \in\mathcal{J}$, and $A_1,\ldots,A_k$ are pairwise disjoint (\textit{finite superadditivity}).
                   \newline
       
             c) $\BP(A)\leq \sum_{n} \BP(A_n)$ whenever $A,A_1,A_2,\ldots \in\mathcal{J}$, and $A\subseteq \bigcup_{n} A_n$ (\textit{countable monotonicity}).               
             \newline
             
             Then there is a $\sigma$-algebra $\mathcal{M} \supseteq \mathcal{J}$ and a probability measure $\BP^*$ on $\mathcal{M}$ such that $\BP^*(A)=\BP(A)$ for all $A\in \mathcal{J}$.
             \end{Theorem}    
       
                
                 \end{itemize}
}

\subsubsection{Review From Last Lecture }
\frame
{
  \frametitle{Constructing Probability Triples I }

   \begin{itemize}

      
       \item[] <1->  Goal:  constructing complicated probability triples on sample space $\Omega$
       
              \item <1->[1)]  Select $\mathcal{J}$, a collection of subsets of $\Omega$ that forms a \textit{semialgebra}. A semialgebra includes empty set and sample space, closed under finite intersections, and the complement of a set in $\mathcal{J}$ can be represented as the unions of disjoint sets from $\mathcal{J}$.
               \item <1->[2)]  Define a function $\mathbf{P}$ from $\mathcal{J}$ to $[0,1]$ that is finitely superadditive, countably monotonic and $P(\emptyset)=0, P(\Omega)=1$.
               \item <2-> [3)]   Construct outer measure $\BP^*$ over all subsets of $\Omega$ based on $P$:
               
    $$\BP^*(A)=\inf_{A_1,A_2\ldots \in \mathcal{J}, A\subseteq \bigcup_i A_i} \sum_i \BP(A_i) $$

               We have shown that the outer measure is an extension of $\mathbf{P}$, is monotonic and countably subadditive over *all* subsets.
               
              
                 \end{itemize}
}

\frame
{
  \frametitle{Constructing Probability Triples II }

   \begin{itemize}

      
        \item <1->[4)]   We constructed a new collection of subsets, $\mathcal{M}$:
 
                         $$\mathcal{M}:=\{A: A\subseteq \Omega, \BP^*(A\cap E)+\BP^*(A^c\cap E)=\BP^*(E) \text{ for all }E\subseteq \Omega \}$$

 We have shown that $\mathcal{M}$ includes the empty set and sample space, and is closed under complement. And based on this definition, it is easy to see that $\BP^*(A)=1-\BP^*(A^c)$ for all $A\in\mathcal{M}$. 
 
        \item <2->[5)]  To verify that $\mathcal{M}$ is a $\sigma$-algebra and $\BP^*$ is a proper probability measure on $\mathcal{M}$, we still need to show that $\mathcal{M}$ is closed under countable unions and $\BP^*$ is countably additive on $\mathcal{M}$. 
        
        
        
 
            
           \item<3->[6)]  Note that, if we need to verify that whether a given set $A\in \mathcal{M}$. By the countable subadditivity of outer measure, we always have $\BP^*(E) \leq \BP^*(A\cap E)+\BP^*(A^c\cap E)$ for all $E\subseteq \Omega$. Thus we only need to verify  $\BP^*(E) \geq \BP^*(A\cap E)+\BP^*(A^c\cap E)$ for all $E\subseteq \Omega$. 
           
       
                 \end{itemize}
}





\subsubsection{$\BP^*$ is Countably Additive over $\mathcal{M}$}
\frame
{
  \frametitle{$\BP^*$ is Countably Additive over $\mathcal{M}$}

   \begin{itemize}

            \item<1->   [] \begin{Lemma}[2.3.9] For disjoint $A_1, A_2,\ldots \in \mathcal{M}$, $\BP^*(\bigcup_n A_n)=\sum_n \BP^*(A_n)$.               \end{Lemma}    
       
              \item<2-> []\textbf{Proof:} We will show the finite additivity first.               
             \item<3-> [1)]  For disjoint $A_1, A_2 \in \mathcal{M}$, since $A_1 \in \mathcal{M}$, we have (in the definition of $\mathcal{M}$, let $E=A_1\cup A_2$, and $A=A_1$):
             
             $\BP^*(A_1\cup A_2)=\BP^*(A_1\cap (A_1\cup A_2))+\BP^*(A_1^c \cap (A_1\cup A_2))=\BP^*(A_1)+\BP^*(A_2)$
             
             The finite additivity would then follow as the result of induction. 
             
                            \item<4-> [2)] For a countably disjoint sequence, by the finite additivity and monotonicity of $\BP^*$, we have
                            
                            $\sum_{i=1}^n \BP^*(A_i)=\BP^*(\bigcup_{i=1}^n A_i)\leq \BP^*(\bigcup_n A_n)$
                            
Furthermore,       $\sum_n \BP^*(A_n)=\lim_{n\rightarrow \infty} \sum_{i=1}^n \BP^*(A_i) \leq \BP^*(\bigcup_n A_n)$
                            
                \item<5-> [3)]  By the countable subadditivity of $\BP^*$, we have  $\sum_n \BP^*(A_n) \geq \BP^*(\bigcup_n A_n)$, thus  $\sum_n \BP^*(A_n) = \BP^*(\bigcup_n A_n)$ for disjoint sequence of $\mathcal{M}$.
    
                 \end{itemize}
}


\subsubsection{$\mathcal{M}$ is Closed under Finite Intersections/Unions}
\frame
{
  \frametitle{$\mathcal{M}$ is Closed under Finite Intersections/Unions}

   \begin{itemize}

            \item<1->   [] \begin{Lemma}[2.3.10] If $A_1, A_2,\ldots, A_n \in \mathcal{M}$, then $\bigcap_{i=1}^n A_i \in \mathcal{M}$ and  $\bigcup_{i=1}^n A_i \in \mathcal{M}$.                \end{Lemma}    
       
              \item<2-> []\textbf{Proof:} Since $\mathcal{M}$ is closed under complement, then by de Morgan's law, $\mathcal{M}$ is closed under finite unions if $\mathcal{M}$ is closed under finite intersections. So we need to show if $A, B\in \mathcal{M}$, then $A\cap B\in \mathcal{M}$.
                         
             \item<3-> [1)]  For any $E\subseteq \Omega$, 
             \begin{align*}& \BP^*(A\cap B \cap E)+\BP^*( (A\cap B)^c \cap E) \\ = & \BP^*(A\cap B \cap E)+\BP^*( (A^c\cap B \cap E) \cup (A\cap B^c \cap E) \cup (A^c\cap B^c \cap E) )  \\ \leq & \BP^*(A\cap B \cap E)+\BP^*( A^c\cap B \cap E) \\ + & \BP^*(A\cap B^c \cap E) +\BP^*( A^c\cap B^c \cap E) \\= & \BP^*(B \cap E)+\BP^*( B^c \cap E)=\BP^*(E)
             \end{align*}
                 
              \item<4-> [2)]  By subadditivity, $\BP^*(E) \leq \BP^*(A\cap B \cap E)+\BP^*( (A\cap B)^c \cap E)$. Thus $\BP^*(E)=\BP^*(A\cap B \cap E)+\BP^*( (A\cap B)^c \cap E)$ and we have $A\cap B\in \mathcal{M}$.
             
             
                 \end{itemize}
}




\subsubsection{$\mathcal{M}$ is Closed under *Countable* Unions of *Disjoint* Sets}
\frame
{
  \frametitle{$\mathcal{M}$ is Closed under *Countable* Unions of *Disjoint* Sets: I}

   \begin{itemize}

            \item<1->  To show that $\mathcal{M}$ is closed under countable unions of disjoint sets, we need the following result
\begin{Lemma}[2.3.11] Let $A_1, A_2,\ldots \in \mathcal{M}$ be disjoint. Define $B_n=\bigcup_{i=1}^n A_i$, then for any $E\subseteq \Omega$, we have 
             $\BP^*(E\cap B_n)=\sum_{i=1}^n \BP^* (E\cap A_i)$. \end{Lemma}    
       
              \item<2-> []\textbf{Proof:} Since $B_n\in \mathcal{M}$ for all $n\in \mathbf{N}$, and note that $B_{n-1}\cap B_n=B_{n-1}$ and $B_{n-1}^c\cap B_n=A_{n}$, we have:
             \begin{align*} \BP^*( E\cap B_n)& = \BP^*( B_{n-1}\cap E\cap B_n)+ \BP^*( B_{n-1}^c \cap E\cap B_n) \\ 
             &= \BP^*( E\cap B_{n-1})+ \BP^*(  E\cap A_n)             \end{align*}
                 
              \item<3-> [] It is obvious that $\BP^*( E\cap B_1)=\BP^*( E\cap A_1)$, then the above equation would allow us to use induction to obtain $\BP^*(E\cap B_n)=\sum_{i=1}^n \BP^* (E\cap A_i)$.
             
             
                 \end{itemize}
}


\frame
{
  \frametitle{$\mathcal{M}$ is Closed under *Countable* Unions of *Disjoint* Sets: II}

   \begin{itemize}

            \item<1->   [] \begin{Lemma}[2.3.13] For disjoint $A_1, A_2,\ldots \in \mathcal{M}$,  $\bigcup_{n} A_n\in \mathcal{M}$.                \end{Lemma}    
       
              \item<2-> []\textbf{Proof}: Let $B_n=\bigcup_{i=1}^n A_i$, then for any $E\subseteq \Omega$
$$ \BP^*( E) =   \BP^*( E\cap B_n)+ \BP^*(E\cap B_n^c)  
             = \sum_{i=1}^n \BP^*( E\cap A_{i})+ \BP^*(E\cap B_n^c)  $$
                $$   \geq \sum_{i=1}^n \BP^*( E\cap A_{i})+ \BP^*(  E\cap (\bigcup_{j=1}^{\infty} A_j) ^c) $$
                   
                   \item<3-> [] Letting $n \to \infty$, we have: 
                     $\BP^*( E) \geq \sum_{n} \BP^*( E\cap A_{n})+ \BP^*(  E\cap (\bigcup_{n} A_n) ^c)   \geq  \BP^*( E\cap (\bigcup_n A_{n}) )+ \BP^*(  E\cap (\bigcup_{n} A_n) ^c)$. Thus $(\bigcup_{n} A_n) \in \mathcal{M}$
   

\end{itemize}
}

\subsubsection{$\mathcal{M}$ is a $\sigma$-algebra}
\frame
{
  \frametitle{$\mathcal{M}$ is a $\sigma$-algebra}

   \begin{itemize}

            \item<1->   [] \begin{Lemma}[2.3.14] $\mathcal{M}$ is a $\sigma$-algebra  \end{Lemma}    
       
              \item<2-> []\textbf{Proof}: We only need to prove that $\mathcal{M}$ is closed under *countable* *not disjoint* unions. Let $A_1, A_2,\ldots \in \mathcal{M}$. For each $n$, define $B_n=A_n\bigcap (\bigcup_{i=1}^{n-1} A_i)^c$.                    
                   \item<3-> [] Since we already show that $\mathcal{M}$ is closed under complement, finite intersections/unions, $B_n\in \mathcal{M}$.
                   \item<4-> [] As $B_n\bigcap B_m=\emptyset$ for all $n\neq m$, by the previously established result, $\bigcup_n B_n \in \mathcal{M}$. However, $\bigcup_n A_n=\bigcup_n B_n$, so $\bigcup_n A_n \in \mathcal{M}$. Then $\mathcal{M}$ is closed under countable unions.
\end{itemize}
}

\subsubsection{$\mathcal{J} \subseteq \mathcal{M}$}
\frame
{
  \frametitle{$\mathcal{J} \subseteq \mathcal{M}$}

\begin{itemize}

\item<1->   [] \begin{Lemma} $\mathcal{J}\subseteq \mathcal{M}$ \end{Lemma}    
       
\item<2-> []\textbf{Proof}: We need to show that for any $A\in \mathcal{J}$, $\BP^*(A\cap E)+\BP^*(A^c\cap E)\leq \BP^*(E)$ for all $E \subseteq \Omega$. 
              
\item<3->  By the definition of outer measure and A.4.2, for any $\varepsilon>0$, we can find $B_1, B_2, \ldots \in\mathcal{J}$ such that $E\subseteq \bigcup_n B_n$ and $\sum_n \BP(B_n) < \BP^*(E)+\varepsilon$. Furthermore, by the definition of semialgebra, $A^c=\bigcup_{k=1}^K J_k$ where $J_1,J_2,\ldots, J_k\in \mathcal{J} $ are pairwise disjoint. Thus: 

\item<4->$ \BP^*(E\cap A)+\BP^*(E\cap A^c ) \leq \BP^*((\bigcup_n B_n)\cap A)+\BP^*((\bigcup_n B_n) \cap A^c) $

\item<5->[] $=\BP^*( \bigcup_n (B_n \cap A) )+\BP^*((\bigcup_n B_n)\cap ( \bigcup_{k=1}^K J_k) )$ 

\item<6->[]  $\leq \sum_n \BP^*( B_n \cap A)+\sum_{n,k}\BP^*(B_n \cap  J_k)=\sum_n \BP( B_n \cap A)+\sum_{n,k}\BP(B_n \cap  J_k) $

\item<7->[]  $= \sum_n \left\{ \BP( B_n \cap A)+\sum_{k}\BP(B_n \cap  J_k) \right\}\leq \sum_n \BP( B_n)  < \BP^*(E)+\varepsilon $

\item<8-> As $\varepsilon$ is an arbitrary constant, we  can conclude that $\BP^*(A\cap E)+\BP^*(A^c\cap E)\leq \BP^*(E)$. Thus $A\in \mathcal{M}$, $\mathcal{J}\subseteq \mathcal{M}$ .
                                      

                   \end{itemize}
}


\end{document}
