%\documentclass{beamer}
\documentclass[handout]{beamer}
%\usepackage{beamerthemesplit} // Activate for custom appearance
\usetheme{Boadilla}
\newtheorem{Proposition}[theorem]{Proposition}%
\newcommand{\BP}{\mathbf{P}}
\newcommand{\BE}{\mathbf{E}}
\newcommand{\BI}{\mathbf{1}}
\newcommand{\BV}{\mathbf{Var}}



\setbeamertemplate{theorems}[numbered]

\title{STAT 7200}
 \subtitle{Introduction to Advanced Probability \newline Lecture 21}
\author{Taylor R. Brown}
\institute{}
\date{}

\begin{document}

\frame{\titlepage}


\section[Outline]{}
\frame{\tableofcontents}

\section{(Levy's) Continuity Theorem} 


\frame
{
  \frametitle{(Levy's) Continuity Theorem} 

 \begin{itemize}
 
\item<1->[] \begin{Theorem}[The Continuity Theorem (11.1.14)] Let $\mu, \mu_1, \mu_2,\ldots$  be probability measures with characteristic functions $\phi, \phi_1, \phi_2,\ldots$. Then $\mu_n$ converges weakly to $\mu$ if and only if $\phi_n(t)\rightarrow \phi(t)$ for all $t \in \mathbf{R}$.  That is, the weak convergence is equivalent to the pointwise convergence of characteristic functions. \end{Theorem}

\item<2-> \textbf{Proof:} \textit{(1) Weak convergence implies pointwise convergence of characteristic functions:}
\item<3->[-] Since $cos(x)$ and $sin(x)$ are both bounded and continuous functions, then for any $t\in \mathbf{R}:$
\begin{align*}
\phi_n(t)& =\int cos(tx) \mu_n(dx)+i \int sin(tx)\mu_n dx \\
&\rightarrow \int cos(tx) \mu (dx)+i \int sin(tx) \mu(dx)=\phi(t),
\end{align*}

\noindent by the definition of weak convergence. 

\end{itemize}
}



\frame
{
  \frametitle{Continuity Theorem: Pointwise Convergence of Characteristic Function Implies Weak Convergence} 

 \begin{itemize}
 
 \item<1-> \textbf{Proof:} (2) On the other hand, if we have $\phi_n(t)\rightarrow \phi(t)$ for all $t \in \mathbf{R}$, we do not even know if the limit of $\{\mu_n\}$ exists or not. 
 
 \item<2-> We will need several theorems, lemmas and corollaries to show that this is indeed true. Many of these will need their own results to prove them.
 
% \item<2->[-] Now suppose that a subsequence $\{\mu_{n_k}\}$ converges weakly to a measure $\nu$,  then by part (1), $\phi_{n_k}(t)\rightarrow \phi_{\nu}(t)$ for all $t$, which suggests  
% $\phi_{\nu}(t)= \phi(t)$. Does this result imply $\nu=\mu$?
% 
% \item<3>[Q1:]  \textbf{Does a characteristic function uniquely determine a probability measure?} 
% 
% \item<4-> [-] A "yes" answer to question 1 implies that, if a subsequence of $\{\mu_n\}$ converges weakly to a probability measure, then this limit must be $\mu$. In this sense, $\mu$ is the only possible weak limit of $\{\mu_n\}$. So to establish $\mu_n$ converges weakly to $\mu$, we need to answer the second question: 
% 
%  \item<5>[Q2:] \textbf{Under what condition, if $\mu$ is the only possible weak limit of a sequence of probability measures, this sequence of probability measures must converge weakly to  $\mu$?} (Answer: $\{\mu_n\}$ must be ``tight".)
% 
 
 
\end{itemize}
}


\frame
{
  \frametitle{Fourier Inversion Theorem} 

 \begin{itemize}
 
  \item<1->[] \begin{Theorem}[Fourier Inversion Theorem (11.1.1)] Let $\mu$ be a Borel probability measure on $\mathbf{R}$ with characteristic function $\phi(t)=\int e^{itx} \mu(dx)$. Then for $a<b$ and $\mu(\{a\})=\mu(\{b\})=0$:
  
  $$\mu([a,b])=\lim_{T\rightarrow \infty} \frac{1}{2\pi} \int_{-T}^T \frac{e^{-ita}-e^{-itb}}{it} \phi(t) dt.$$
  
   \end{Theorem}
  
 
 \item<2-> \textbf{Remark 1:} We'll prove this with two Lemmas.
 
 \item<3-> \textbf{Remark 2:} The number of intervals $[a,b]$ with $\mu(\{a\}) \neq 0 $ or $\mu(\{b\})\neq 0$ is at most countable because the set $\{x:\mu(\{x\})>0\}$ is at most countable. That's from the previous lecture.  
 
\end{itemize}
}


\frame
{
  \frametitle{First Lemma to prove Fourier Inversion Theorem} 

 \begin{itemize}
 
  \item<1->[] \begin{Theorem}[Lemma 11.1.2] 
  For $T \ge 0$ and $a < b$
  	$$\int_{\mathbf{R}} \int_{-T}^T \left| \frac{ e^{-ita} - e^{-itb} }{it} e^{itx} \right| dt \mu(dx) \le 2T(b-a) < \infty.$$
   \end{Theorem}
  
 
\end{itemize}
}


\frame
{
  \frametitle{First Lemma to prove Fourier Inversion Theorem} 

\begin{align*}
\left| \frac{ e^{-ita} - e^{-itb} }{it} e^{itx} \right| &= \left| \frac{ e^{-ita} - e^{-itb} }{it}  \right||e^{itx}| \\
&= \left| \int_a^b e^{itr} dr  \right| \\
&\le  \int_a^b |e^{itr}| dr  \\
&= b-a 
\end{align*}

So
$$
\int_{\mathbf{R}} \int_{-T}^T \left| \frac{ e^{-ita} - e^{-itb} }{it} e^{itx} \right| dt \mu(dx) \le \int_{\mathbf{R}} \int_{-T}^T (b-a) dt \mu(dx) = 2T(b-a)
$$

}


\frame
{
  \frametitle{Second Lemma to prove Fourier Inversion Theorem} 

 \begin{itemize}
 
  \item<1->[] \begin{Theorem}[Lemma 11.1.3] 
	For $T \ge 0$ and $\theta \in \mathbf{R}$
	$$
	\lim_{T \to \infty}\int_{-T}^T \frac{\sin(\theta t)}{t}dt = \pi \text{sign}(\theta)
	$$
	where $\text{sign}(\theta)$ is either $1$, $-1$ or $0$ depending on whether $\theta$ is positive, negative or $0$, respectively.
   \end{Theorem}
  
We're omitting the proof because it's elementary integration, but it's fun and involves a lot of cool stuff (e.g. the sinc function, integration by parts, u-substitution, different trigonometric properties, etc.) so you should try it. It's also in the book.
\end{itemize}
}



\frame
{
  \frametitle{Fourier Inversion Theorem: Proof} 

WTS: $$\mu([a,b])=\lim_{T\rightarrow \infty} \frac{1}{2\pi} \int_{-T}^T \frac{e^{-ita}-e^{-itb}}{it} \phi(t) dt$$

\begin{align*}
&\frac{1}{2\pi} \int_{-T}^T \frac{e^{-ita}-e^{-itb}}{it} \phi(t) dt \\
&= \frac{1}{2\pi} \int_{-T}^T \frac{e^{-ita}-e^{-itb}}{it} \left( \int_{\mathbf{R}} e^{itx} \mu(dx)  \right) dt  \\
&= \frac{1}{2\pi} \int_{\mathbf{R}} \int_{-T}^T \frac{ e^{it(x-a)}-  e^{it(x-b)}  }{it} dt  \mu(dx) \tag{Fubini and first Lemma}\\ 
&= \frac{1}{2\pi} \int_{\mathbf{R}} \int_{-T}^T \frac{ i\sin(t(x-a))- i \sin( t(x-b))  }{it} dt  \mu(dx)  
\end{align*}  
}

\frame
{
  \frametitle{Fourier Inversion Theorem: Proof (continued) } 
Taking $T \to \infty$:
\begin{align*}
& \lim_{T \to \infty} \frac{1}{2\pi} \int_{-T}^T \frac{e^{-ita}-e^{-itb}}{it} \phi(t) dt \\
&= \lim_{T \to \infty}\frac{1}{2\pi} \int_{\mathbf{R}} \int_{-T}^T \frac{ \sin(t(x-a))-  \sin( t(x-b))  }{t} dt  \mu(dx)  \\
&= \frac{1}{2\pi}   \int_{\mathbf{R}} \lim_{T \to \infty} \int_{-T}^T \frac{ \sin(t(x-a))-  \sin( t(x-b))  }{t} dt  \mu(dx)  \tag{DCT} \\
&= \frac{1}{2\pi}   \int_{\mathbf{R}} \pi\left[\text{sign}(x-a) - \text{sign}(x-b) \right]  \mu(dx)  \tag{Lemma 2} \\
&= \mu((a,b)) + \frac{1}{2} \mu(\{a\}) + \frac{1}{2}\mu(\{b\}) \\
&=\mu([a,b])
\end{align*}  
where the last equality follows because $\mu(\{a\})=\mu(\{b\})=0$.	

}



\frame
{
  \frametitle{Fourier Uniqueness Theorem: Characteristic Function Determines Distribution} 

 \begin{itemize}
 
  \item<1->[] \begin{Theorem}[Fourier Uniqueness Theorem] Let $X, Y$ be random variables. Then $\phi_X(t)=\phi_Y(t)$ if and only if $\mathcal{L}(X)=\mathcal{L} (Y)$.   
   \end{Theorem}
  
  \item<2->\textbf{Proof:} The ``if" part is trivial. For the ``only if'' part, to show $\mathcal{L}(X)=\mathcal{L} (Y)$, we only need to show $\BP(X\in I)=\BP(Y\in I)$ for all intervals in $\mathbf{R}$ (uniqueness of extension Prop. 2.5.8). 
  
  \item<3->[-] By Fourier Inversion theorem, $\BP(X\in [a,b])=\BP(Y\in [a,b])$ whenever $a<b$ and $\BP(X=a)=\BP(X=b)=\BP(Y=a)=\BP(Y=b)=0$. 
  \item<4->[-] For any interval $I$, we can always find a sequence of closed intervals $\{[a_i, b_i]\}$ that satisfy the above conditions, and $[a_i,b_i]\rightarrow I$. Thus, we can apply the continuity of probability to show that $\BP(X\in I)=\BP(Y\in I)$.
  
  \end{itemize}
}



%\frame
%{
%  \frametitle{Fourier Inversion Theorem: Proof} 
%
% \begin{itemize}
% \item<1->[] $$\mu([a,b])=\lim_{T\rightarrow \infty} \frac{1}{2\pi} \int_{-T}^T \frac{e^{-ita}-e^{-itb}}{it} \phi(t) dt.$$
%  \item<1-> \textbf{Proof:} First, we have:
%    \begin{equation}\frac{1}{2\pi} \int_{-T}^T \frac{e^{-ita}-e^{-itb}}{it} \phi(t) dt=\frac{1}{2\pi} \int_{-T}^T \frac{e^{-ita}-e^{-itb}}{it} (\int_{\mathbf{R}} e^{itx} \mu(dx) ) dt \label{fourier1}\end{equation}
% 
% \item<2->[-] We claim that we may exchange the order of integrals in (\ref{fourier1}), since
% 
% $$|\frac{e^{-ita}-e^{-itb}}{it}  e^{itx}|=|\int_a^b e^{-itr} dr|\leq \int_a^b |e^{-itr}| dr=b-a$$
% 
%So $\int_{\mathbf{R}}\int_{-T}^T | \frac{e^{-ita}-e^{-itb}}{it}  e^{itx}|   dt \mu(dx) \leq 2T(b-a)<\infty$. 
%
%\item<3->[-]Then we may apply Fubini's theorem on (\ref{fourier1}) and:
%  \begin{equation}\frac{1}{2\pi} \int_{-T}^T \frac{e^{-ita}-e^{-itb}}{it} \phi(t) dt=\frac{1}{2\pi} \int_{\mathbf{R}} \int_{-T}^T \frac{e^{it(x-a)}-e^{it(x-b)}}{it}  dt \mu(dx)\label{fourier2}\end{equation}
%   
%\end{itemize}
%}
%
%
%
%\frame
%{
%  \frametitle{Fourier Inversion Theorem: continued} 
%
% \begin{itemize}
% \item<1->\textbf{continued:} As $e^{itc}=\cos(tc)+i\sin(tc)$, and $\cos(tc)/(it)$ is an odd function, then $\int_{-T}^T e^{itc} /(it) dt=\int_{-T}^T \sin(tc)/t dt$, thus:
%\begin{equation}  \resizebox{0.9\hsize}{!} {$\frac{1}{2\pi} \int_{\mathbf{R}} \int_{-T}^T \frac{e^{it(x-a)}-e^{it(x-b)}}{it} dt \mu(dx) = \frac{1}{2\pi} \int_{\mathbf{R}} \int_{-T}^T \frac{\sin[t(x-a)]-\sin[t(x-b)]}{t} dt \mu(dx)  $} \label{fourier3}\end{equation}
%
% \item<3->[-] Utilize the fact that, for any $T\geq 0$ and $\theta\in \mathbf{R}$, there is a constant $C$:
% 
% $$\lim_{T\rightarrow \infty} \int_{-T}^T\frac{\sin \theta t}{t} dt =\pi \text{sign} (\theta),  \text{and } |\int_{-T}^T\frac{\sin \theta t}{t} dt|\leq C.$$
% 
%  \item<4->[-] We can then let $T\rightarrow \infty$ in (\ref{fourier3}), and exchange limit and the integral (bounded convergence theorem):
%\scalebox{0.9}{$ \begin{array}{ll} &   \lim_{T\rightarrow \infty} \frac{1}{2\pi} \int_{\mathbf{R}} \int_{-T}^T \frac{e^{it(x-a)}-e^{it(x-b)}}{it} dt \mu(dx)  \\ & = \frac{1}{2\pi} \int_{\mathbf{R}} \lim_{T\rightarrow \infty}  \int_{-T}^T \frac{\sin[t(x-a)]-\sin[ t(x-b)]}{t} dt \mu(dx)  \\ &
%= \frac{1}{2\pi} \int_{\mathbf{R}} \pi [\text{sign}(x-a)-\text{sign}(x-b)] \mu(dx)=\frac{1}{2} \mu(\{a\})+\mu((a,b))+\frac{1}{2}\mu(\{b\}) \end{array}
%$}
%
% \end{itemize}
%}



\frame
{
\frametitle{Helly Selection Principle} 

\begin{Theorem}[Helly Selection Principle] 
Let $\{F_n\}$ be a sequence of cdfs, each corresponding with a measure $\mu_n$. Then there exists a subsequence $F_{n_k}$, and a non-decreasing, right-continuous $0 \le F \le 1$, such that $F_{n_k}(x) \to F(x)$ for each $x \in \mathbf{R}$ that is a continuity point of $F$.
\end{Theorem}

Proof: a lot of Bolzano-Weierstrass.
\newline

Also, $F$ is not necessarily a cdf.
}

\frame
{
\frametitle{Helly Selection Principle: proof} 

List out rationals $\mathbf{Q} = \{q_1, q_2,\ldots\}$. Note that $0 \le F_n(q_1) \le 1$ for all $n$. By Bolzano-Weirstrass, there exists a subsequence $l_k^{(1)}$ such that $\lim_k F_{l_k^{(1)} }(q_1)$ exists. Then there exists a subsequence of that subsequence, call it $l_k^{(2)}$, such that $\lim_k F_{l_k^{(2)} }(q_2)$ exists. Because it is a subsequence of the first one, we also have that $\lim_k F_{l_k^{(2)} }(q_1)$ exists. We can do this for each $m \in \mathbf{N}$. For each $\{l_{k}^{(m)} \}$, we have $\lim_k F_{l_k^{(m)} }(q_j)$ exists for all $0 < j \le m$.
\newline

Now define $n_k = l_k^{(k)}$. These are the diagonals. However, note that all these subsequences are nested, so for any $k$, 
\begin{itemize}
\item $\{n_k, n_{k+1}, \ldots\} \subseteq \{l^{(k)}_{k}, l^{(k)}_{k+1},\ldots\}  $ 
\item $\{n_{k+1}, n_{k+2}, \ldots\} \subseteq \{l^{(k+1)}_{k+1}, l^{(k+1)}_{k+2},\ldots\}  $, etc. 
\end{itemize}

These ensure that $\lim_k F_{n_k}(q) := G(q)$ exists for each $q \in \mathbf{Q}$. $G$ is also clearly non-decreasing as well.

}

\frame
{
\frametitle{Helly Selection Principle:proof} 

For each rational $q$, $F_{n_k}(q) \to  G(q)$. $G$ is defined on the rationals, only. Now we define 
$$
F(x) = \inf\{ G(q) : q > x, q \in \mathbf{Q} \}
$$
which is defined on the reals. It has a few properties that we'll need:

\begin{itemize}
\item $F$ is non-decreasing
\item $0 \le F \le 1$
\item $F$ is right-continuous,and
\item $F(q) \ge G(q)$ for all $q \in \mathbf{Q}$
\end{itemize}

Next, we'll show that, for any continuity point of $F$, call it $x \in \mathbf{R}$, we have $F_{n_k}(x) \to F(x)$ as $k \to \infty$. Pick any $\varepsilon > 0$, then pick $r,u,s \in \mathbf{Q}$ such that $r < u < x < s$ and $F(s) - F(r) < \varepsilon$.

\begin{align*}
F(x) - \varepsilon &\le F(r) \\
&= \inf\{ G(q) : q > r\} \\
&= \inf\{ \lim_{k} F_{n_k}(q) : q > r\} 
\end{align*}
}


\frame
{
\frametitle{Helly Selection Principle:proof} 


\begin{align*}
F(x) - \varepsilon &\le \inf_{q > r} \liminf_{k} F_{n_k}(q) \\
&\le \liminf_{k} F_{n_k}(u) \tag{$u > r$} \\
&\le \liminf_{k} F_{n_k}(x) \tag{$x > u$ } \\
&\le \limsup_{k} F_{n_k}(x)  \\
&\le \limsup_{k} F_{n_k}(s) \tag{$s > x$ } \\
&= G(s) \\
&\le F(s) \\
&\le F(x) + \varepsilon 
\end{align*}
QED
}



\frame
{
 \frametitle{Tightness of Measure}

\begin{itemize}
\item<1-> For a sequence of cdfs $F_{n}$, there exists a convergent subsequence. However, the limit, $F$, isn't necessarily a cdf. $\lim_{x \to \infty} F(x) < 1$, for example. 

\item<2-> We need to introduce the concept of tightness.

\item<3-> \textbf{Tightness of Measure} A collection of probability measure $\{\mu_n\}$ on $\mathbf{R}$ is \textbf{tight} if for all $\varepsilon>0$, there are $a<b$ with $\mu_n([a,b]) \geq 1-\varepsilon$ for all $n$. (Probability mass does not ``escape of infinity").

\item<4-> \textbf{Example:} 1) $\{N(0, \frac{1}{n})\}$ is tight. 2) $\{N(0, n)\}$ is not tight.

\item<5-> \textbf{Property:} Any subsequence of a tight sequence is tight.  \end{itemize}
}


\frame
{
  \frametitle{Tightness of Measure: Three More Results }

\begin{theorem}[11.1.10]
 If $\{\mu_n\}$ is a tight sequence of prob. measures, then there exists a subsequence $\{\mu_{n_k}\}$ and a probability measure $\mu$ such that $\mu_{n_k}$ converges weakly to $\mu$. 
 \end{theorem}
 
\begin{theorem}[Corollary 11.1.11]
 Let $\{\mu_n\}$ be a tight sequence of prob. measures on $\mathbf{R}$. Also suppose that, whenever $\mu_{n_k} \Rightarrow \nu$, then $\nu$ is always equal to $\mu$. Then $\mu_n \Rightarrow \mu$.
 \end{theorem}

\begin{Lemma}[11.1.13]
 Let $\{\mu_n\}$ be a sequence of probability measures on $\mathbf{R}$, and $\{\phi_n(t)\}$ be the characteristic functions. If there is a function $g$ that is continuous at 0, and $\phi_n(t)\rightarrow g(t)$ for all $|t|<t_0$ ($t_0>0$), then $\{\mu_n\}$ is tight. 
 \end{Lemma}

Remember our goal is to prove the other direction of Levy's continuity theorem. Let's prove the third one first.

}


% TODO prove these

\frame
{
  \frametitle{Tightness and Characteristic Functions}

 \begin{itemize}
 
 \item<1->[] \begin{Lemma}[11.1.13]
 Let $\{\mu_n\}$ be a sequence of probability measures on $\mathbf{R}$, and $\{\phi_n(t)\}$ be the characteristic functions. If there is a function $g$ that is continuous at 0, and $\phi_n(t)\rightarrow g(t)$ for all $|t|<t_0$ ($t_0>0$), then $\{\mu_n\}$ is tight. 
 \end{Lemma}
 
 \item<2-> \textbf{Proof:} 
 Let $y > 0$
% We will show that for any $\varepsilon>0$, we can find a particular $y_0$, such that $\mu_n( (-\frac{2}{y_0},\frac{2}{y_0}))>1-\varepsilon$ for all $n$. 
% \item<3->[-] For this purpose, we will first investigate the following integral:
\begin{align*} 
\frac{1}{y} \int_{-y}^{y} [1-\phi_n(t)] dt
& = \int_{-\infty}^{\infty} \left[ \frac{1}{y} \int_{-y}^{y} (1-e^{itx}) dt \right]\mu_n (dx)  \\
&= 2 \int_{-\infty}^{\infty} (1-\frac{\sin y x}{y x}) \mu_n (dx)  \\
&\geq 2\int (1-\frac{1}{|y x|}) \mu_n (dx)  \\ 
&\geq \int_{|x| > 2/y} 1 \mu_n (dx)  = \mu_n\left( \left\{x: |x| \geq \frac{2}{y} \right\} \right)  
\end{align*} 


\end{itemize}
 }



\frame
{
  \frametitle{Tightness and Characteristic Functions: continued}

 \begin{itemize}
 
 \item<1-> \textbf{Proof continued:} The previous discussion shows that $\mu_n[\{x: |x| \geq \frac{2}{y}\}] \leq \frac{1}{y} \int_{-y}^{y} [1-\phi_n(t)] dt$. 
 
\item<2->[-] Now since $g(t)$ is continuous at 0, and $g(0)=\lim_n \phi_n(0)=1$, then for any $\varepsilon>0$, we can always find $y_0 \in (0,t_0)$ such that:
$$
|1 - g(t)| < \varepsilon / 4
$$
whenever $|t| < y_0$. Then

\begin{align*}
\left| \frac{1}{y_0} \int_{-y_0}^{y_0} [1-g(t)] dt \right| &\le  \frac{1}{y_0} \int_{-y_0}^{y_0} \left|1-g(t)\right| dt \\
&\le  \frac{1}{y_0} \int_{-y_0}^{y_0} \varepsilon/4 dt = \varepsilon/2 \\
\end{align*}
\end{itemize}
 }
 
 \frame
{
  \frametitle{Tightness and Characteristic Functions: continued}

 \begin{itemize}
 
 \item<1-> \textbf{Proof continued:} The previous discussion shows that 
 \begin{enumerate}
 \item $\mu_n[\{x: |x| \geq \frac{2}{y}\}] \leq \frac{1}{y} \int_{-y}^{y} [1-\phi_n(t)] dt$ and
 \item $\left| \frac{1}{y_0} \int_{-y_0}^{y_0} [1-g(t)] dt \right| \le \varepsilon/2$
 \end{enumerate} 
 
\item<3->[-] On the other hand, as $\phi_n(t)\rightarrow g(t)$ for  $|t|<t_0$, and $|\phi_n(t)|=1$, we can apply the dominated convergence theorem:
 $$ \left|\int_{-y_0}^{y_0} [1-\phi_n(t)] dt  -\int_{-y_0}^{y_0} [1-g(t)] dt \right| \leq \varepsilon/2 $$
  for $n>N$. 
\item<4->[-] Then for all $n>N$, 
$$\mu_n[\{x: |x| \geq \frac{2}{y_0}\}] \leq \frac{1}{y_0} \int_{-y_0}^{y_0} [1-\phi_n(t)] dt \leq \varepsilon$$
\item<5->[-] It then follows that $\{\mu_n\}$ must be tight. 

\end{itemize}
 }
 
 
 \end{document}
