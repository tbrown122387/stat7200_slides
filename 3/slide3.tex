%\documentclass{beamer}
\documentclass[handout]{beamer}
% \usepackage{beamerthemesplit} // Activate for custom appearance
\usetheme{Boadilla}
\newtheorem{Proposition}[theorem]{Proposition}

\setbeamertemplate{theorems}[numbered]

\title{STAT 7200}
 \subtitle{Introduction to Advanced Probability \newline Lecture 3}
\author{Taylor R. Brown}
\institute{}
\date{}

\begin{document}

\frame{\titlepage}

\section[Outline]{}
\frame{\tableofcontents
``A First Look at Rigorous Probability Theory" (Jeffrey Rosenthal)  A.5, Section 1
}



\section{Mathematical Background}




\subsection{Equivalence Relations} 

\frame
{
  \frametitle{Equivalence Relations}

   \begin{itemize}
   
 
\item<1-> \textbf{Equivalence Relations} We can define relations on a set $S$ such that for any $x,y\in S$, either $x\sim y$ (related) or  $x\not \sim y$ (not related). A relation $\sim$ is an {\bf equivalence relation} if (a) it is reflexive, $x\sim x$ for all $x\in S$; (b) it is symmetric, $x\sim y$ whenever $y\sim x$; and (c) it is transitive, $x\sim z$ if $x\sim y$ and $y\sim z$.
  
\item<2-> We use equivalence relations to partition up a state space.
       
\item<3->  \textbf{Example } Let $\mathbf{N}$ be the set of all natural number. For $m, n \in \mathbf{N}$,  $m \sim n$ if and only if $|m-n|$ is an integer multiple of 2.  Show that: 1) This is an equivalence relation. 2) This equivalence relation leads to the partition of $\mathbf{N}$ as the union of the set of even numbers and the set of odd numbers.        
  \end{itemize}
}

\section{Not Every Subset Has a Probability!}
\subsubsection{What is a ``Reasonable" Probability Measure?} 

\frame
{
  \frametitle{What is a ``Reasonable" Probability Measure? }

   \begin{itemize}
   
 
  \item<1-> \textbf{Uniform Measure over [0,1]}  Let $X\sim Unif ([0,1])$, a uniform measure over [0,1]. Then one question is, for any subset $A\subseteq [0,1]$, can we define the probability $P(A)=P(X\in A)$?
  
 \item<2-> First, as [0,1] contains all outcomes, we expect $P([0,1])=1$. 

 
 
 \item<3-> Second, as the word ``uniform" suggests, each $x$ has an ``equal chance" to be a realization of $X$. We would then conclude the probability that $X$ equals any particular number is $0$. This ``uniform" principle also implies that, for any set $A$ of the form $[a,b], (a,b), [a,b), or (a, b]$, we have $P( A)=b-a$.
 
 
\item<4-> Third, if we ask what is the probability that the value of $X$ falls into one or more disjoint intervals, naturally the answer should be the sum of probabilities of each intervals. This leads to the requirement known as ``finite additivity":
  
$P(A\cup B)=P(A)+P(B)$ when $A \cap B = \emptyset$
  \end{itemize}
}

\frame
{
  \frametitle{What is a ``Reasonable" Probability Measure?: Continued }

   \begin{itemize}
   
 
  
 \item<1-> Fourth, we would like to extend  ``finite additivity" to ``countable additivity" to account for the problems like:
 
 $1=P([0,1])=P({0}\cup [1,\frac{1}{2}]\cup (\frac{1}{2}, \frac{1}{4}]\cup (\frac{1}{4},\frac{1}{8}] \cup \cdots )$

 
 
 \item<2-> Fifth, however, we could not expect ``uncountable additivity." If this was true: we would encounter blatant paradoxes: $1=P([0,1])=P(\cup_{x \in [0,1]} \{x\}) =\sum_{x\in [0,1]} P(\{x\})=0$.
 
  \item<3-> Sixth, the ``uniform" would also suggest that if we ``shift" any subset of [0,1] by a fixed amount, the probability of the shifted subset should equal the probability of the original subset. That is, for $A\subseteq [0,1]$ and $0 \le r < 1$, if we define its $r-shift$ as $A\oplus r=\{a+r: a\in A, a+r\leq 1\} \cup\{a+r-1: a\in A, a+r> 1\} $. Then $P(A\oplus r)=P(A)$

  
  \item<4-> However, it turns out that there is a contradiction if we assume {\bf all} sets get a probability. 
  
    \end{itemize}
}




\subsubsection{A Proof By Contradiction}

\frame
{
  \frametitle{A Proof By Contradiction}

   \begin{itemize}

      
   \item[]<1-> \begin{Proposition}[1.2.6] Let $A \subseteq [0,1]$. It is impossible to define $P(A)$ that satisfies:  
   
   1) for interval $A$ in the forms of $[a,b], (a,b), [a,b),(a, b]$, $P( A)=b-a$; 
   
   2) countable additivity: $P(\cup_{n=1}^{\infty} A_n)=\sum_{n=1}^{\infty} P(A_n)$ if all $A_n$ are disjoint. 
   
   3) $P(A\oplus r)=P(A)$. \end{Proposition} 
      \item<2->\textbf{Proof} Assume to the contrary that it is possible to define such a probability measure. Here we define an equivalence relation on [0,1]: $x\sim y$ if and only if  $x-y$ is a rational number. This equivalence relation would induce a partition on [0,1]. 
      
      \item<3-> By the Axiom of Choice, define $H \subseteq [0,1]$ by picking exactly one element from each disjoint subset in the partition. We assume $0\not\in H$ as we can always replace it with $0.5$ ($0\sim 0.5$). 
       
  \end{itemize}
}


\frame
{
  \frametitle{A Limitation of Probability Measure: continued}

   \begin{itemize}

      
       \item<1->\textbf{Proof: continued:}  Let $r_1, r_2 \in [0,1) \cap \mathbf{Q}$. For $r_1\neq r_2$, $H\oplus r_1$ and $H\oplus r_2$ are disjoint. If not, we can find $x\in [0,1]$ such that $x\in H\oplus r_1$ and $x\in H\oplus r_2$. Then there are $h_1\neq  h_2 \in H$ such that $x=h_1\oplus r_1=h_2\oplus r_2$. As a result, $x\sim h_1$ and  $x\sim h_2$, and we must accept that $h_1\sim h_2$. However, this contradicts the definition of $H$. 
       
      \item<2->It is also easy to check that $[0,1]=\cup_{r \in \mathbf{Q} \cap [0,1) } (H\oplus r)]$ since each element in $[0,1]$ must be equivalent to an element in $H$. 
      
      \item<3->Finally, by {\bf countable} additivity: $1=P((0,1])=P(\cup_{r \in \mathbf{Q} \cap [0,1] } (H\oplus r) )=\sum_{r \in \mathbf{Q} \cap [0,1) } P(H\oplus r)$. 
      
      That is: $1=\sum_{r \in \mathbf{Q} \cap [0,1) } P(H)$. 
      
      \item<4-> However, this is impossible! The sum must either be $0$ or $\infty$, but never 1. 
             
  \end{itemize}
}


\section{Probability Triple}
\frame
{
  \frametitle{Probability Triple}

   \begin{itemize}

      
       \item<1->\textbf{Probability Triple}  All probability models are built on a triple $(\Omega,\mathcal{F}, P)$, the sample space, $\sigma$-algebra and probability measure. 
       
       
             \item<2-> $\Omega$ is the sample space, the set of all possible outcomes. 
             \item<3-> $\mathcal{F}$ is a collection of measurable subsets of $\Omega$, a $\sigma$-algebra. For a collection $\mathcal{F}$ to be qualified as a  $\sigma$-algebra , the following conditions must be met: 
            
            a) $\emptyset \in \mathcal{F}$;    b) If $A \in \mathcal{F}$ then $A^c\in \mathcal{F}$.
            
            c) If $A_1, A_2,\ldots\in \mathcal{F}$ implies $\bigcup_{n=1}^{\infty} A_n \in \mathcal{F}$.
            
             We can prove that a $\sigma$-algebra must also closed under countable intersection, finite union and finite intersection. 
             
              \item<4-> $P: \mathcal{F} \to [0,1]$ is the probability measure. It must satisfy the following axioms: a)  $P(A)\geq 0$ for any $A\in \mathcal{F}$. b) $P(\Omega)=1$; c) Countable additivity. 
              
              More properties regarding probability measure can be found in section 2.1. 
                \end{itemize}
}




\subsection{Probability Triples with Finite or Countable Sample Spaces}
\frame
{
  \frametitle{Probability Triples with Finite or Countable Sample Spaces}

   \begin{itemize}

      
       \item<1-> Suppose the sample space is countable or finite: $\Omega=\{\omega_1, \omega_2,\ldots\}$. Then we may define probability measure in the following way:  
       

             \item<2-> $\mathcal{F}$ is chosen  as the power set, the collection of all possible subsets of $\Omega$. 
             
                     \item<3-> Suppose $\{a_1, a_2, \ldots\}$ is any sequence with the same cardinality as $\Omega$. Suppose further that $0\leq a_i \leq 1, \sum_i a_i=1$. Then we can define the probability measure as:
                     
                    $$P(\omega_i)=a_i \text{ for all } \omega_i \in \Omega$$
                     
                     $$P(A)=\sum_{i: \omega_i \in A} a_i$$
                                          
            
             
              \item<4-> It is easy to check that the above definition leads to a proper probability triple: $\mathcal{F}$ is a $\sigma$-algebra and the probability measure $P$ defined  satisfies all the axioms of probability. 
                              \end{itemize}
}



\subsection{Probability Triple with Uncountable Sample Space}
\frame
{
  \frametitle{Probability Triple with Uncountable Sample Space}

   \begin{itemize}

      
            \item<1-> We have already discussed that it is impossible to define probability measure on {\bf all} the possible subsets. Still, we expect that the $\sigma$-algebra should contain certain subsets, such as intervals. 
            
               \item<2-> \textbf{Borel $\sigma$-algebra} can be defined as the smallest $\sigma$-algebra that contains all the open intervals $(a,b)$.  
               
                \item<3-> According to the fact that $\sigma$-algebra is closed under complements, unions and intersections (both finite and countable), we can show that the Borel $\sigma$-algebra must also contain other intervals as well: ($\emptyset, \{a\}, [a,b], (a,b], [a,b), [a,\infty), (a,\infty),  (-\infty,a), (-\infty,a] $)
               
                             \item<4-> For instance, [a,b] must be a Borel Set since $[a,b]=(a,b) \cup \cap_{n=1}^\infty (a-\frac{1}{n}, a+\frac{1}{n} ) \cup \cap_{n=1}^\infty (b-\frac{1}{n}, b+\frac{1}{n} )$                                               \end{itemize}
}


% \subsection{Borel $\sigma$-algebra}
% \frame
% {
%   \frametitle{Borel $\sigma$-algebra}
% 
%    \begin{itemize}
% 
%       
%        \item<1->  To see how complicated a Borel set can be, let us see the following cases: 
%        
%             \item<2-> 1) Any open set on $\mathbf{R}$ is a Borel Set. 
%             \item[]<3-> To see why this is true, recall that if $A$ is an open set is, for any $x\in A$, we may find an open interval containing $x$ and is also a subset of $A$. Then for any $x\in A$, denote $I_x=(a_x, b_x)$ such that $x \in I_x, I_x\subseteq A$ and $a_x, b_x \in \mathbf{Q}$. And we can rewrite $A=\cup_{x\in A} I_x$. However, since $I_x$ is an open interval with both ends as rational numbers, there can only be countable many different $I_x$. Thus, $\cup_{x\in A} I_x$ is essentially a countable union. And open set $A$ is a Borel Set. 
%             
%             \item<4-> 2) Any closed set on $\mathbf{R}$ is a Borel Set. 
%             \item[]<5->  A closed set is defined as the complement of a open set, so a close set should be a Borel set. 
%             
%                  \end{itemize}
% }
% 
% \frame
% {
%   \frametitle{Borel $\sigma$-algebra: continued}
% 
%    \begin{itemize}
% 
%       
%        \item<1->  Now let $\mathcal{G}_1$ consist of all open sets,  $\mathcal{F}_1$ consist of all the closed sets. A union of open sets is always open, but a union of closed sets may not be closed (for instance, $\cup_n [a,b-\frac{1}{n}]=[a,b)$ is not a closed set). 
%        
%         \item<2->  Let $\mathcal{G}_2$ consist of all countable unions of closed sets. But the complement of such a set may not yet have been accounted for (i.e  the complement of the union of rational numbers ). So we construct a new collection $\mathcal{F}_2$ that consists of all complements of sets in $\mathcal{G}_2$.
%     
%             \item<3->  Continuing in this way, we can define Borel sets of ever increasing complexity (by taking unions or complements of sets generated earlier). However, such process does not stop after any finite number of steps (even after infinitely many steps, you would still not exhaust all Borel sets).
%            
%             \item<4->  Thus, it is impractical to define probability measure over Borel $\sigma$-algebra as we define probability measure over the discrete  space. Instead, we will rely on the so-called extension theorem, such that probability measure can be defined over a smaller collection first, and then uniquely extend to a larger set. 
%                
%                  \end{itemize}
% }


\subsection{Probability Triple with Uncountable Sample Space}
\frame
{
  \frametitle{Getting ready for next class}


   \begin{itemize}
   \item<1-> $\mathcal{J} := \{\text{ empty, singleton, closed, open, right/ left closed (all) intervals in } [0,1] \}$    
   \item<2-> $\mathcal{B}_0 := \{ \text{all finite unions of elements from } \mathcal{J} \}$
   \item<3-> $\mathcal{B}_1 := \{ \text{all finite or countable unions of elements from } \mathcal{J} \}$
   \end{itemize}
   
   Try showing
   \begin{itemize}
   \item Exercise 2.2.3.: Prove $\mathcal{J}$ is a semialgebra.
   \item Exercise 2.2.5: Prove $\mathcal{B}_0$ is an algebra, but not a $\sigma$-algebra.
   \item Exercise 2.4.7: Prove $\mathcal{B}_1$ is still not a $\sigma$-algebra.
   \end{itemize}
}


\end{document}
